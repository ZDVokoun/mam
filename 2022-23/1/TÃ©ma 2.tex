\documentclass{fkssolpub}

\usepackage[czech]{babel}
\usepackage{fontspec}

\usepackage{fkssugar}
 
\author{Ondřej Sedláček}
\school{Gymnázium Oty Pavla} 
\series{1}
\problem{2} 

\begin{document} 

\section{Úloha 1}

Experiment jsem prováděl ve svém pokoji, kde, když jsem nic nepoustěl, 
bylo něco málo pod 30 dB. Jako reproduktor jsem použil JBL Charge 3,
který ale špatně hraje nízké tóny, a na vytvoření uzavřeného prostoru
jsem použil ručník a malou kartónovou krabici.

Postup byl následující. Pro každý tón s frekvencemi od 55 Hz, 110 Hz až
po 3520 Hz jsem nejdříve změřil jejich hlasitost bez izolace, pak jsem
pro stejné tóny měřil hlasitost, když byl reproduktor zabalen do ručníku
a vložen do krabice. 

Ale otázkou zůstává, jak vypočítáme ten koeficient absorpce. Známe tyto
vztahy:

\[
  k = \frac{I_{\text{absorbováno}}}{I_{\text{celková}}}
\]
\[
  L = 10 \times \log{\frac{I}{I_0}}
\]

Při odvozování intenzity $I$ využiji toho, že $I_0$ je konstanta,
tudíž při dosazení do prvního vzorce neovlivní výsledek:

\[
  L = 10 \times \log{\frac{I}{I_0}}
\]
\[
  \log{\frac{I}{I_0}} = \frac{L}{10}
\]
\[
  \frac{I}{I_0} = 10^{\frac{L}{10}}
\]

Označím-li hladinu intenzity bez izolace $L_1$ a hladinu intenzity 
s izolací $L_2$, dostaneme pro $k$ tento vzorec:

\[
  k = \frac{10^{\frac{L_1}{10}} - 10^{\frac{L_2}{10}}}{10^{\frac{L_1}{10}}}
\]

Zde jsou výsledky dvou měření, kdy u druhé jsem měl zapnutou vyšší
hlasitost:

\begin{table}[h!]
  \caption{Výsledky prvního měření}
  \label{tab:1}
  \begin{center}
    \begin{tabular}{|c|c|c|c|c|}
      \hline
      Číslování & $f$ & $L_1$ & $L_2$ & $k$ \\
      \hline
      1 & 55 & 38 & 38 & 0 \\
      2 & 110 & 60 & 58 & 0.369042655519807 \\
      3 & 220 & 61 & 59 & 0.369042655519806 \\
      4 & 440 & 66 & 60 & 0.748811356849042 \\
      5 & 880 & 63 & 54 & 0.874107458820583 \\
      6 & 1760 & 57 & 40 & 0.980047376850311 \\
      7 & 3520 & 50 & 30 & 0.99 \\
      \hline
    \end{tabular}
  \end{center}
\end{table}

\begin{table}[h!]
  \caption{Výsledky druhého měření}
  \label{tab:2}
  \begin{center}
    \begin{tabular}{|c|c|c|c|c|}
      \hline
      Číslování & $f$ & $L_1$ & $L_2$ & $k$ \\
      \hline
      1 & 55 & 44 & 39 & 0.683772233983162 \\
      2 & 110 & 64 & 58 & 0.748811356849042 \\
      3 & 220 & 71 & 59 & 0.936904265551981 \\
      4 & 440 & 70 & 53 & 0.980047376850311 \\
      5 & 880 & 71 & 54 & 0.980047376850311 \\
      6 & 1760 & 65 & 43 & 0.993690426555198 \\
      7 & 3520 & 58 & 32 & 0.99748811356849 \\
      \hline
    \end{tabular}
  \end{center}
\end{table}

Pro zprůměrované výsledky viz tabulku \ref{tab:cel}.

\begin{table}[h!]
  \caption{Zprůměrování výsledků}
  \label{tab:cel}
  \begin{center}
    \begin{tabular}{|c|c|c|c|c|}
      \hline
      Číslování & $f$ & $k_1$ & $k_2$ & $\prumer k$ \\
      \hline
        1 & 55 & 0 & 0.683772233983162 & 0.341886116991581 \\
        2 & 110 & 0.369042655519807 & 0.748811356849042 & 0.558927006184425 \\
        3 & 220 & 0.369042655519806 & 0.936904265551981 & 0.652973460535893 \\
        4 & 440 & 0.748811356849042 & 0.980047376850311 & 0.864429366849677 \\
        5 & 880 & 0.874107458820583 & 0.980047376850311 & 0.927077417835447 \\
        6 & 1760 & 0.980047376850311 & 0.993690426555198 & 0.986868901702755 \\
        7 & 3520 & 0.99 & 0.99748811356849 & 0.993744056784245 \\
      \hline
    \end{tabular}
  \end{center}
\end{table}


Na měření ale nalézám dvě zvláštnosti, a to že se naměřená absorpce mění
tolik s rostoucí hlasitostí a také že ten koeficient je takhle velký.
To první dokážu vysvětlit jen tím, že při nižší hlasitosti se více promítá
hluk prostředí, tudíž to ty výsledky zmátlo. A to druhé se vysvětluje tím,
že intenzita roste vůči hladiny intenzity exponenciálně.

\section{Problém 2}

Změnou prostředí bychom mohli dojít k jiným výsledkům díky jinému množství
hluku nebo díky menšímu množství odražených vln, které můžou zvuk zesilovat
a tím teoreticky zkreslit výsledky. Výsledky taky může ovlivňovat technika,
například chytrý telefon nemusel přesně vyhodnotit hladinu intenzity, nebo
reproduktor špatně hrál určité rozmezí tónů. Též mohlo dojít k softwarové
chybě. Výsledky mohli ovlivnit také způsob izolování reproduktoru nebo
hlasitost reproduktoru.

\section{Úloha 3}

Protože musíme sečíst intenzity a ne hladinu intenzit, musíme nejdříve získat
intenzitu z hladiny intenzity:

\[
  L = 10 \times \log{\frac{I}{I_0}} \ztoho \frac{I}{I_0} = 10^{\frac{L}{10}}
\]

Tohle nám stačí znát, abychom mohli vypočítat hladinu intenzity zvuku dvou vysavačů:

\[
  L = 10 \times \log{2 \times 10^{\frac{L}{10}}} = 10 \times \log{2 \times 10^7}
    = 73.01 dB
\]

\section{Úloha 4}

Hluk bych definoval jako jakýkoli zvuk, který je nám nežádoucí. Do toho zcela jistě
zapadá veškeré bolestivé zvuky. Nebo třeba zvuky, které odvádějí pozornost, jako
hlas rušícího kolegy, notifikace, hluk z ulice, plačící dítě, zvuk vrtačky nebo
dupání. 


\end{document}
