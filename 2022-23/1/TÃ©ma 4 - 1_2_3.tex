\documentclass{fkssolpub}

\usepackage[czech]{babel}
\usepackage{fontspec}

\usepackage{fkssugar}
 
\author{Ondřej Sedláček}
\school{Gymnázium Oty Pavla} 
\series{1}
\problem{4} 

\begin{document} 

\section{Úloha 1}

Tohle lze ukázat následujícím způsobem:
\[
  (a \times f)' = a' \times f + a \times f' = 0 \times f + a \times f' = a \times f'
\]

\section{Úloha 2}

Nejdříve si to ukážeme na menších číslech. Pro $x^2$ lze derivaci odvodit takto:
\[
  (x^2)' = (x \times x)' = 2x \times x' = 2x
\]

A derivaci $x^3$ takto:
\[
  (x^3)' = (x^2 \times x) = x^2 \times x' + x \times (x^2)' = x^2 + 2x^2 = 3x^2
\]

Můžeme si zde všimnout toho, že po použití vzorce na součin nám zbyde zderivovat
mocnina s exponentem nižší o jedna, což napovídá, že by to mohl být rekurentní
vzorec, což se ukáže jako pravda:

\[
  (x^n)' = (x \times x^{n-1})' = x^{n-1} + x \times (x^{n-1})'
\]

Ale co s tím můžeme dělat? Zkusme to ještě rozložit:
\[
  x^{n-1} + x \times (x^{n-1})' = x^{n-1} + x \times (x^{n-2} + x \times (x^{n-2})')
    = 2x^{n-1} + x^2 \times (x^{n-2})'
\]

A zde můžeme vidět, že počet $x^{n-1}$ a exponent čísla násobící funkci, která zbývá
derivovat, je stejný jako počet kol derivace. Tudíž když provedeme $n-1$ kol, vyjde:
\[
  (x^n)' = (n-1) \times x^{n-1} + x^{n-1} \times x' = n \times x^{n-1}
\]

To jsme chtěli dokázat.

\section{Problém 3}

Vzorec pro derivaci podílů funkcí lze odvodit takto:
\[
  \left(\frac{f}{g}\right)' = (f \times g^{-1})' = f' \times g^{-1} + f \times (g^{-1})'
\]

Než budeme pokračovat, musím definovat $h(x) = x^{-1}$:
\[
  f' \times g^{-1} + f \times (g^{-1})' = f' \times g^{-1} + f \times (h(g))'
    = f' \times g^{-1} + f \times (h'(g) \times g') 
    = f' \times g^{-1} - f \times g^{-2} \times g' 
    = \frac{f' \times g - f \times g'}{g^2} 
\]

A hotovo.


\end{document}
