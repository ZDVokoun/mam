\documentclass{fkssolpub}

\usepackage[czech]{babel}
\usepackage{fontspec}

\usepackage{fkssugar}
 
\author{Ondřej Sedláček}
\school{Gymnázium Oty Pavla} 
\series{1}
\problem{4} 

\begin{document} 

Jako první budu derivovat tento výraz:
\begin{equation}
  5x^4-3x^2+\pi
  \label{eq:1}
\end{equation}

Na tento výraz použijeme jenom vzorce pro součet a odčítání, 
součin, konstantu a umocněné číslo:
\[
  (5x^4-3x^2+\pi)' = 5(x^4)' - 3(x^2)' = 20x^3 - 6x
\]

Jako další budu derivovat toto:
\begin{equation}
  \cos^2 x + \sin^2 x
  \label{eq:2}
\end{equation}

Tady budeme sčítance derivovat samostatně. Začneme u prvního členu.

Jedná se o složenou funkci kosinu a kvadratické funkce. Pro pohodlí
je zde budu označovat následovně:
\[
  f(x) = x^2
\]
\[
  g(x) = \cos x
\]

Pak je dosadíme a získáme derivaci prvního sčítance:
\[
  (f(g))' = f'(g) \times g' = -2 \sin \times \cos
\]

Druhý sčítanec zderivujeme podobně:
\[
  f(x) = x^2
\]
\[
  g(x) = \sin x
\]
\[
  (f(g))' = f'(g) \times g' = 2 \sin \times \cos
\]

Teď do výrazu \ref{eq:2} dosadíme tyto derivace:
\[
  (\cos^2 x + \sin^2 x)' = 2 \sin \times \cos - 2 \sin \times \cos = 0
\]

Další zderivujeme druhou odmocninu z x:
\begin{equation}
  \sqrt{x}
  \label{eq:3}
\end{equation}

Druhá odmocnina je inverzní funkcí ke kvadratické rovnici $x^2$.
Proto když označíme $f(x) = x^2$, můžeme ji zderivovat:
\[
  (f^{-1})'(x) = \frac{1}{f'(f^{-1}(x))} = \frac{1}{2\sqrt{x}}
\]

Dále zderivujeme přirozenou exponenciální funkci:
\begin{equation}
  e^{-3x}
  \label{eq:4}
\end{equation}

Jedná se o složenou funkci, proto označíme:
\[
  f(x) = e^x
\]
\[
  g(x) = -3x
\]

Teď ji zderivujeme:
\[
  (f(g(x)))' = f'(g(x)) \times g'(x) = -3 \times e^{-3x}
\]

Jako další máme výraz v podílovém tvaru:
\begin{equation}
  \frac{3x^3 + 42x}{7x^2 + 2x}
  \label{eq:5}
\end{equation}

Tento výraz nejprve zjednodušíme:
\[
  \frac{3x^3 + 42x}{7x^2 + 2x} = \frac{3x^2 + 42}{7x + 2}
\]

Pak její členy označíme jako:
\[
  f(x) = 3x^2 + 42
\]
\[
  g(x) = 7x + 2
\]

A zderivujeme:
\[
  f'(x) = (3x^2 + 42)' = 3(x^2)' = 6x
\]
\[
  g'(x) = (7x + 2)' = 7
\]

A pak můžeme derivovat:
\[
  \left(\frac{f}{g}\right)' = \frac{f'g - fg'}{g^2} = 
  \frac{6x (7x + 2) - 7 (3x^2 + 42)}{(7x + 2)^2} =
  \frac{42x^2 + 12x - 21x^2 - 296}{(7x + 2)^2} =
  \frac{21x^2 + 12x - 296}{(7x + 2)^2} = 
  \frac{3(7x^2 + 4x - 98)}{(7x + 2)^2}
\]

Teď musíme získat tvar bez derivací tohoto výrazu:
\begin{equation}
  (\sin(5x))''
  \label{eq:6}
\end{equation}

Při první derivaci využijeme vzorec pro složené funkce
a funkci sinus a při druhé ještě pro kosinus:
\[
  (\sin(5x))' = 5 \sin'(5x) = 5 \cos(5x)
\]
\[
  (\sin(5x))'' = (5 \cos(5x))' = 5 (\cos(5x))' = -25 \sin(5x)
\]

A jako poslední nám zbývá tento výraz:
\begin{equation}
  \ln(\tan x)
  \label{eq:7}
\end{equation}

Pro něj využijeme vzorce pro přirozený logaritmus a tangens
z článku a pro složenou funkci:
\[
  (\ln(\tan x))' = \ln'(\tan x) \times \tan' x = 
  \frac{1}{\tan x} \times \frac{1}{\cos^2 x} = 
  \frac{1}{\tan x \times \cos^2 x}
\]


\end{document}
