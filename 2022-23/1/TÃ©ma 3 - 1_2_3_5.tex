\documentclass{fkssolpub}

\usepackage[czech]{babel}
\usepackage{fontspec}

\usepackage{fkssugar}

\usepackage{tikz}
\usetikzlibrary{arrows,automata,positioning}
 
\author{Ondřej Sedláček}
\school{Gymnázium Oty Pavla} 
\series{1}
\problem{3} 

\begin{document} 

\section{Problém 1}

Aby bylo používání výtahu bezpečné, musí ještě sledovat, jestli mezi dveřmi
není nějaký překážející objekt, také po zmáčknutí tlačítka musí čekat, nesmí 
umožnit změnu směru uprostřed jízdy atd.

\section{Problém 2}

Výtah bude potřebovat senzory na vnímání polohy výtahy, zaznamenání
objektu překážejícího před zavřením. Dále bude potřebovat dva páry tlačítek,
jedny v patrech a druhé ve výtahu. Na dvojici tlačítek pro stejné patro stroj
reaguje stejně.

\begin{center}
\begin{tikzpicture}[shorten >=1pt,node distance=4cm,on grid,auto]
  \tikzstyle{every state}=[fill={rgb:black,1;white,10}]

  \node[state,initial] (priz-otevr) {otevřeno v přízemí};
  % \node[state] (rozhl-otevr) [below left of=priz-otevr] {otevřeno na rozhledně};
  \node[state] (nahoru) [below left of=priz-otevr] {jede nahoru};
  \node[state] (priz-zavir) [below right of=priz-otevr] {zavírá v přízemí};
  \node[state] (rozhl-zavir) [below of=nahoru] {zavírá na rozhledně};
  % \node[state] (nahoru) [below of=priz-zavir] {jede nahoru};
  \node[state] (rozhl-otevr) [below of=priz-zavir] {otevřeno na rozhledně};
  \node[state] (dolu) [below right of=rozhl-zavir] {jede dolů};

  \path[->]
  (priz-otevr) edge [bend left, align=center] node {1} (priz-zavir)
  (priz-zavir) edge [bend left, align=center] node {2} (nahoru)
               edge [bend left, align=center] node {3} (priz-otevr)
  (nahoru) edge [bend right, align=center] node {4} (rozhl-otevr)
           edge [loop above, align=center] node {5} ()
  (rozhl-otevr) edge [bend left, align=center] node {6} (rozhl-zavir)
  (rozhl-zavir) edge [bend right, align=center] node {7} (dolu)
                edge [bend left, align=center] node {8} (rozhl-otevr)
  (dolu) edge [bend left, align=center] node {9} (priz-otevr)
         edge [loop below, align=center] node {10} ();

\end{tikzpicture}
\end{center}

\subsection{Vysvětlivky}

\begin{enumerate}
  \item zmáčknuto tlačítko na rozhlednu / počkej  5s a zavři dveře v přízemí
  \item dveře nezablokovány / jeď nahoru
  \item dveře zablokovány \textbf{NEBO}  zmáčknuto tlačítko  k přízemí / otevři  dveře v přízemí 
  \item dojel na rozhlednu / otevři dveře na rozhledně
  \item zmáčknuto tlačítko na rozhlednu \textbf{NEBO} zmáčknuto  tlačítko k přízemí / jeď nahoru
  \item zmáčknuto tlačítko k přízemí / počkej 5s a zavři dveře v přízemí
  \item dveře nezablokovány / jeď dolů
  \item dveře zablokovány \textbf{NEBO}  zmáčknuto tlačítko  na rozhlednu / otevři  dveře v přízemí
  \item dojel k přízemí / otevři dveře v přízemí
  \item zmáčknuto tlačítko na rozhlednu \textbf{NEBO} zmáčknuto  tlačítko k přízemí / jeď dolů
\end{enumerate}

\section{Úloha 3}

Na funkčnost tohoto výtahu jsou potřeba čtyři páry tlačítek -- jeden z páru
je ve výtahu a druhý se nachází v patře, do kterého výtah přivolává. Tento 
Mealyho stroj se řídí podle fronty, kam při zmáčknutí tlačítka pro patro $n$
se přidá číslo $n$. Stroj pak z ní číslo $n$ smaže, jakmile tam dorazí, nebo 
když už tam je:

\begin{tikzpicture}[shorten >=1pt,node distance=6cm,on grid,auto]
  \tikzstyle{every state}=[fill={rgb:black,1;white,10}]
  % \draw [help lines] (-1,1) grid (3,-1);

  \node[state,initial] (otevreno) {otevřeno v $n$};
  \node[state] (jede) [below right of=otevreno] {jede do $n$};
  \node[state] (zavreno) [above right of=jede] {zavřeno};

  \path[->]
  % (otevreno) edge [loop above, align=center] node {ve frontě aktuální patro $n$ / otevřít \\ v $n$ patře a odstranit z fronty} ( )
  %            edge [bend left, align=center] node {ve frontě jiné patro $n$ / zavření v aktuálním patře \\ a přesun do patra $n$ \\
  %                                   \textbf{NEBO} \\
  %                                   fronta prázdná / zavření v aktuálním patře} (zavreno)
  % (zavreno) edge [loop right, align=center] node {ve frontě jiné patro $n$ / přesun do patra $n$} ( )
  %           edge [bend left, align=center] node {dorazil do patra $n$ / otevřít v $n$ patře \\ a odstranit z fronty \\
  %                                   \textbf{NEBO} \\
  %                                   ve frontě aktuální patro $n$ / otevřít v $n$ patře a odstranit z fronty} (otevreno);

  (otevreno) edge [loop above, align=center] node {ve frontě aktuální patro $n$ / otevřít \\ v $n$ patře a odstranit z fronty} ( )
             edge [bend left, align=center] node {ve frontě jiné patro $n$ / zavření v aktuálním patře \\
                                    \textbf{NEBO} \\
                                    fronta prázdná / zavření v aktuálním patře} (zavreno)
  (zavreno) edge [bend left, align=center] node {ve frontě jiné patro $n$ / přesun do patra $n$} (jede)
            edge [bend left, align=center] node {ve frontě aktuální patro $n$ / otevřít v $n$ patře \\ a odstranit z fronty} (otevreno)
            edge [loop right, align=center] node {fronta prázdná / zavřít \\ v patře $n$} ()
  (jede) edge [bend left, align=center] node {dorazil do patra $n$ / otevřít \\ v $n$ patře a odstranit z fronty} (otevreno)
         edge [loop below, align=center] node {} ();

\end{tikzpicture}

\section{Úloha 5}

Abychom na žádný přechod nezapomněli, níže je tabulka všech 
kombinací s údajem, který udává, jestli se při manuálním resetu
změní konečný stav daného přechodu.

\begin{center}
  \begin{tabular}{|c|c|c|c|}
    \hline
    počáteční stav & koncentrace CO & konečný stav & jiný stav při manuálním resetu \\
    \hline
    klid & CO > 50 ppm & poplach & ne \\
    poplach & CO > 50 ppm & poplach & ne \\
    klid & CO < 5 ppm & klid & ne \\
    poplach & CO < 5 ppm & klid & ne \\
    klid & 5 ppm $\leq$ CO $\leq$ 50 ppm & klid & ne \\
    poplach & 5ppm $\leq$ CO $\leq$ 50 ppm & poplach & ano \\
    \hline
  \end{tabular}
\end{center}

Teď můžeme dokreslit všechny potřebné přechody:


\begin{tikzpicture}[shorten >=1pt,node distance=8cm,on grid,auto]
  \tikzstyle{every state}=[fill={rgb:black,1;white,10}]

  \node [state, initial] (klid) {klid};
  \node [state] (poplach) [right of=klid] {poplach};

  \path[->]
  (klid) edge [bend left,align=center] node {CO: > 50 ppm / poplach = True} (poplach)
         edge [loop below,align=center] node {CO: $\leq$ 50 ppm / poplach = False \\
                                    \textbf{NEBO} \\
                                   manuální reset / poplach = False} ()
  (poplach) edge [bend left,align=center] node {CO: < 5 ppm / poplach = False \\
                                    \textbf{NEBO} \\
                                    manuální reset / poplach = False} (klid)
            edge [loop above,align=center] node {CO: $\geq$ 5 ppm / poplach = True} ();
\end{tikzpicture}

\end{document}
