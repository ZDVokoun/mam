\documentclass{fkssolpub}

\usepackage[czech]{babel}
\usepackage{fontspec}

\usepackage{fkssugar}
 
\author{Ondřej Sedláček}
\school{Gymnázium Oty Pavla} 
\series{1}
\problem{1} 

\begin{document} 

\section{Úloha 1}

Jedinou možnou zmínkou o železu je reálné číslo ž. Protože výrok, 
ve kterém se reálné číslo ž nachází, není zcela ideální na práci, 
převedl jsem ho do následujícího tvaru:

\begin{equation}
  \label{eq:kletba}
  \forall \text{ž} \in \mathbb{R}^+ : 
    \text{vzdálenost}_{\text{meč od osoby}} > \text{ž} \land 
    \text{kletba}_{\text{meč}} \not\subset \text{budoucnost}_{\text{osoba}}
\end{equation}

Z tohoto výroku vyplívá, že pokud je osoba dál než reálné číslo ž,
nebude proklet. Tudíž pokud reálné číslo ž označuje množství železa,
pak by množství železa určovalo vzdálenost, na kterou Honzovi nehrozí
prokletí. Tudíž je možnost, že meč je ze železa.

Za kov bych pravděpodobně doporučil bronz, jedná se totiž o nejlepší
náhradu za železo. Je měkčí, ale stále tvrdší než měď nebo zlato.

\section{Úloha 2}

Pokud je zde myšlen negací operátor negace, pak v pohádce se 
nachází pouze dva výroky obsahující negace:
\begin{equation}
  \label{eq:1}
  \{h \in \text{hlavy} : h  \lnot \text{upadla} \} = \emptyset \implies \text{drak } \lnot \text{živý}
\end{equation}

\begin{equation}
  \label{eq:2}
  \lnot \exists h \in \text{hlavy} : h  \lnot \text{upadla}
\end{equation}

Výrok \ref{eq:1} se nabízejí dva způsoby. U jednoho prostě zaměníme
slova tak, že nebude potřeba používat operátor negace, u druhého
budeme muset výrok trochu přepsat.

Při prvním způsobu zaměníme $\lnot \text{upadla}$ za drží a $\lnot \text{živý}$
za mrtvý:

\[
  \{h \in \text{hlavy} : h \text{ drží} \} = \emptyset \implies \text{drak mrtvý}
\]

Při druhém způsobu vyzní výrok jinak než původně, ale hodnoty bude vracet stejné:

\[
  |\{h \in \text{hlavy} : h \text{ upadla} \}| \neq 7 \implies \text{drak živý}
\]

Výrok \ref{eq:2} převedeme jednodušeji, a to následujícím způsobem:

\[
  \forall h \in \text{hlavy} : h \text{ upadla}
\]


\section{Úloha 3}

Na meči stojí:

\begin{center}
  \textit{Pro všechny osoby v království platí, že pokud je jejich budoucnost mečem prokleta, 
budou mít smůlu.}

  \textit{Neexistuje kladné reálné číslo ž, pro které platí, že pokud  
vzdálenost meče od osoby je větší než číslo ž, bude osoba mečem prokleta.}
\end{center}

\section{Úloha 4}

Zde jsem narazil na problém s výroky na meči, protože jediné, co říkají, je to, že
když je člověk proklet, bude mít smůlu, a pokud bude Honza bude od meče dál než je 
velikost kladného reálného čísla ž, nebude proklet. Tudíž můžeme jen spekulovat 
jestli bude Honza proklet či ne.

Pokud by nebyl proklet, pak se ožení s princeznou a stane se králem. Pokud však bude
proklet, pak kdykoliv se mu může něco nemilého stát (např. při návratu zpět zakopne
na hraně útesu a sletí ze skály).
% Vypadalo to, že bude svatba s princeznou a Honza se stane králem, avšak přišla na 
% něj smůla způsobena prokletým mečem, při svatebním obřadu se přepadl přes okno, 
% když se díval do ulic, a zemřel na zranění z pádu.

\section{Úloha 5}

Z textu vyplívá, že Honza je nejhloupější muž v království, protože z jednoho výroku 
vyplívá, že není jiným mužem v království, který má menší nebo stejně velký mozek jako
Honza.

\section{Úloha 6}

Draka lze zabít tak, že mu všechny hlavy upadnou. A zařídit, aby hlava upadla, lze tak, 
že do něj někdo sekne mečem, čímž, jak v textu napsáno, nebude průnik hlavy a meče prázdná
množina. Tudíž se musí do každé jednotlivé hlavy draka seknout mečem, aby umřel.


\end{document}
