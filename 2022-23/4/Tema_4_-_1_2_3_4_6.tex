\documentclass{fkssolpub}

\usepackage[czech]{babel}
\usepackage{fontspec}
\usepackage{fkssugar}

\newcommand{\ceq}{\stackrel{C}{=}}

\author{Ondřej Sedláček}
\school{Gymnázium Oty Pavla} 
\series{4}
\problem{4} 

\begin{document} 

\section{Úloha 1}

U prvního výrazu použijeme vzorec pro konstantu. Při jeho použití vyjde:

\[
  \int 42 \,dx \stackrel{C}{=} 42x
\]

U dalšího použijeme vzorec pro součet a pak pro $\sin x$ a 
$\frac{1}{\cos^2 x}$:

\[
  \int 5 \sin x + \frac{6}{\cos^2 x} \, dx = 5 \int \sin x \, dx 
    + 6 \int \frac{1}{\cos^2 x} \, dx \stackrel{C}{=} -5 \cos x + 6 \tan x
\]

V následujícím použijeme vzorec pro mocniny. Musíme si však dát pozor
na konstanty:

\[
  \int \int 6x \, dx \, dx = \int 3x^2 + C \, dx \stackrel{C}{=}
    x^3 + Cx
\]

Teď jdeme na určité integrály. Ty budu řešit jejich obecným vyřešením a
následným dosazením do definice.

V prvním použijeme vzorec pro mocninu:

\[
  \int x \, dx \ceq \frac{1}{2}x^2
\]
\[
  \int_0^1 x \, dx = \frac{1}{2} \cdot 1^2 + \frac{1}{2} \cdot 0^2
    = \frac{1}{2}
\]

V druhém použijeme vzorec pro $\sin x$:

\[
  \int \sin x \, dx \ceq -\cos x
\]
\[
  \int_0^2\pi = -\cos 2\pi + \cos 0 = 0
\]

V posledním použijeme vzorec, jehož výsledek je $\arcsin x$:

\[
  \int \frac{1}{\sqrt{1 - x^2}} \, dx \ceq \arcsin x
\]
\[
  \int_0^1 \frac{1}{\sqrt{1 - x^2}} \, dx = \arcsin 1 - \arcsin 0
    = \frac{\pi}{2}
\]

\section{Úloha 2}

Nejdříve odvodíme vzorce pro velikost okamžité rychlosti
$v$ a uraženou vzdálenost pro volný pád $s$:

\[
  v = \int g \, dt \ceq gt
\]

\[
  s = \int gt \, dt \ceq \frac{1}{2}g t^2
\]

Protože známe uraženou vzdálenost $s$, můžeme ze vzorce
pro uraženou vzdálenost vypočítat čas volného pádu $t$:

\[
  s = \frac{1}{2} g t^2 \ztoho t = \sqrt{\frac{2s}{g}}
\]
\[
  t = \sqrt{\frac{2 \cdot 58,7}{9,81}} \doteq 3,46 \, \text{s}
\]

Ještě vypočítáme velikost rychlosti $v$ a máme hotovo:

\[
  v = \int_0^t g \, dt = gt = g \cdot \sqrt{\frac{2s}{g}} 
    = \sqrt{2sg} = \sqrt{2 \cdot 9,81 \cdot 58,7} 
    = 33,94 \, \text{m} \cdot \text{s}^{-1}
\]

\section{Úloha 3}

Nejprve výraz upravíme tak, abychom mohli aplikovat vzorec pro složenou
funkci:

\[
  \int \frac{1}{e^x} \, dx = \int e^{-x} \, dx = - \int e^{-x} \cdot (-1) \, dx
\]

Z tohoto tvaru můžeme odvodit, že vnější funkce $F(x) = e^x$ a vnitřní funkce
$G(x) = -x$. Díky tomu lze výraz vyjádřit jako:

\[
  - \int e^{-x} \cdot (-1) \, dx \ceq -e^{-x}
\]

U druhého výrazu budeme muset pro obecné vyjádření také provést menší úpravy:

\[
  \int xe^{-x^2} \, dx = -\frac{1}{2} \cdot \int (-2) xe^{-x^2} \, dx
\]

Zde použijeme pak stejný postup:

\[
  -\frac{1}{2} \cdot \int (-2) xe^{-x^2} \, dx \ceq -\frac{1}{2} e^{-x^2}
\]

Teď můžeme zjistit výsledek určitého integrálu:

\[
  \int_0^2 xe^{-x^2} \, dx = -\frac{1}{2} e^{-2^2} + \frac{1}{2} e^{0^2}
    = \frac{1 - e^{-4}}{2} = \frac{1}{2} - \frac{1}{2e^{4}} \doteq 0,49
\]

\section{Úloha 4}

Na první výraz můžeme krásně použít integraci per partes. Zde je nejlepší
dosadit funkce jako $f(x) = e^x$ a $G(x) = x$. Tím nám vyjde:

\[
  \int x \cdot e^x \, dx = x \cdot e^x - \int e^x \cdot 1 \, dx \ceq e^x (x - 1)
\]

U druhé využijeme nápovědy a dosadíme $f(x) = 1$ a $G(x) = \ln x$:

\[
  \int \ln x \, dx = x \ln x - \int x \cdot \frac{1}{x} \ceq x (\ln x - 1)
\]

U posledního nejdříve musíme využít vzorce $sin^2 x + cos^2 x = 1$, díky níž
spolu s jednou integrací per partes dostaneme rovnost, ze které už lze odvodit
vyjádření integrálu:

\[
  \int \cos^2 x \, dx = \int 1 \, dx - \int \sin^2 x \, dx
    = \int 1 \, dx + \sin x \cos x - \int \cos^2 x \, dx 
\]

\[
  2 \int \cos^2 x \, dx \ceq x + \sin x \cos x
\]

\[
  \int \cos^2 x \, dx \ceq \frac{x + \sin x \cos x}{2}
\]

\section{Úloha 6}

Snad příliš nevadí, že řeším úlohu ze svého článku.
Jestli se Vám bude zdát, že se tímto zvýhodňuji, budu
v pořádku s tím, že tuto úlohu nebudu mít obodovanou.
Teď ale už jdeme na řešení.

Zde nejdříve musíme vyjádřit funkci obvodu závidlou 
na jedné proměnné $f(x)$. Protože známe plochu, můžeme
vyjádřit jednu ze stran pomocí vzorce $S = ab$. Vyjádřením
dostaneme vztah $b = \frac{S}{a} = \frac{800}{a}$. Obvod
plotu v tomto případě bude $o = 2a + b$. Teď už jsme
konečně schopni vyjádřit funkci obvodu:

\[
  f(x) = 2 x + \frac{800}{x} 
    = 2 \left( x + \frac{400}{x} \right)
\]

Následně musíme tuto funkci zderivovat:

\[
  f'(x) = 2 \left( x + \frac{400}{x} \right)'
    = 2 + 2 \frac{0 - 400}{x^2} = 2 - \frac{800}{x^2}
\]

Protože funkce $f(x)$ je kvadratická, jejíž parametr $a$
je kladné číslo, má tato funkce jediný extrém, která je
jejím globálním minimem. Proto funkce $f(x)$ dosáhne
minima, když:

\[
  2 - \frac{800}{x^2} = 0 \ztoho x = \sqrt{400} = 20
\]

Teď už můžeme dopočítat rozměry oplocení:

\[
  a = 20 \, \text{m}
\]
\[
  b = \frac{800}{a} = \frac{800}{20} = 40 \, \text{m}
\]

\end{document}
