\documentclass{fkssolpub}

\usepackage[czech]{babel}
\usepackage{fontspec}
\usepackage{fkssugar}
\usepackage{amsmath}
\newcommand{\ceq}{\stackrel{C}{=}}

\author{Ondřej Sedláček}
\school{Gymnázium Oty Pavla} 
\series{5}
\problem{4} 

\begin{document} 

\section{Úloha 1}

Při postupu provedeme dvakrát per partes:

\[
  \int e^x \sin x \, dx = e^x \sin x - \int e^x \cos x \, dx
\]
\[
  \int e^x \cos x \, dx = e^x \cos x + \int e^x \sin x \, dx
\]

A dosadíme:

\[
  \int e^x \sin x \, dx = e^x \sin x - e^x \cos x - \int e^x \sin x \, dx
\]
\[
  \int e^x \sin x \, dx \ceq e^x \frac{\sin x - \cos x}{2}
\]

\section{Úloha 2}

Nejdříve musíme jednotlivé strany trojúhelníku vyjádřit analyticky. Taková
strana bude ležet na přímce o koeficientech $a$, $b$, která bude splňovat
soustavu:

\[
  ax_1 + b = y_1
\]
\[
  ax_2 + b = y_2
\]

Z ní vyjádřím obecné vzorce pro $a$ a $b$:

\[
  a = \frac{y_1 - y_2}{x_1 - x_2}
\]
\[
  b = y - ax
\]

Stranu ohraničenou body [0,0] a [3,0] nemusíme řešit, protože její funkce je
nulová. U zbytku bude stačit vypočítat určitý integrál.

Nejdříve určíme stranu ohraničenou body [0,0] a [2,1]. Její koeficienty budou:

\[
  a = \frac{1 - 0}{2 - 0} = \frac{1}{2}
\]
\[
  b = 1 - 1 = 0
\]

Výsledek integrálu pak bude:

\[
  \int_0^2 \frac{x}{2} \, dx = \frac{2^2 - 0}{4} = 1
\]

Pak určíme stranu ohraničenou body [2,1] a [3,0]. Její koeficienty budou:

\[
  a = \frac{0 - 1}{3 - 2} = -1
\]
\[
  b = 1 + 2 = 3
\]

Teď určíme další integrál:

\[
  \int_2^3 3 - x \, dx = 3 \cdot 3 - \frac{3^2}{2} - 3 \cdot 2 + \frac{2^2}{2}
    = 3 - \frac{5}{2} = \frac{1}{2}
\]

Když výsledky integrálů sečteme, získáme konečně obsah:

\[
  S = 1 + \frac{1}{2} = \frac{3}{2}
\]

\section{Úloha 3}

Tady se musíme dát pozor na to, abychom správné integrály odečetli, jinak
získáme špatný obsah. Z obrázku získáme následující výraz, který se 
rovná obsahu mnohoúhelníku:

\[
  S = \int_1^2 3 + x \, dx + \int_2^4 5 \, dx + \int_4^5 9 - x \, dx
    - \int_1^3 \frac{11 - 3x}{2} \, dx - \int_3^5 \frac{-7 + 3x}{2} \, dx
\]

Skoro všechny integrály se zde budou upravovat podobně:

\[
  S = \frac{9}{2} + 10 + \frac{9}{2} - 5 - 5 = 9
\]
% \[
%   S = 6 + 2 - 3 - \frac{1}{2} + 20 - 10 + 45 - \frac{25}{2} - 36 + 8
%     - \frac{}{2}
%     - \frac{33 - \frac{27}{2} - 11 + \frac{1}{2}}{2} 
%     - \frac{-35 + \frac{75}{2} + 15 - \frac{27}{2}}{2} = \frac{41}{4}
% \]


\end{document}
