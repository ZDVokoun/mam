\documentclass{fkssolpub}

\usepackage[czech]{babel}
\usepackage{fontspec}
\usepackage{amssymb}

\usepackage{fkssugar}
\usepackage{graphicx}
\usepackage{hyperref}
\usepackage{csquotes}
\usepackage{tabularx}
 
\author{Ondřej Sedláček}
\school{Gymnázium Oty Pavla} 
\series{3}
\problem{2} 

\begin{document} 

\section{Úloha 2}

Pro výpočet jsem si vybral sluchátka JBL Tune 230NC TWS a potřebné informace
na výpočet jsem bral ze stránky: 
\url{https://www.jbl.com/on/demandware.static/-/Sites-masterCatalog_Harman/default/dw2507ba12/pdfs/JBL_Tune_230NCTWS_SpecSheet_USA_English.pdf}.

Důležitý bude především tento údaj:

\begin{displayquote}
Sensitivity: 105dB SPL@1kHz 1mW
\end{displayquote}

Z ní víme ty nejdůležitější informace, ale budeme muset ještě odhadovat
vzdálenost, ve který se to měřilo.

Ale nejdříve musíme odvodit vzorec pro účinnost. Pro reproduktory můžeme určit
účinnost jako poměr akustického výkonu ku elektrickému příkon:

\[
  \eta = \frac{P_a}{P_e}
\]

Protože příkon zjistíme z charakteristické citlivosti, musíme zjistit
akustický výkon. Pro něj platí:

\[
  P_a = \frac{S p^2}{\rho c},
\]

kde $S$ je plocha, na kterou zvuk působí, $p$ je akustický tlak, $\rho$ je
hustota média a $c$ je rychlost zvuku v médiu. Protože se všechna měření
dělají ve vzduchu, $\rho = 1,2 \text{kg} \cdot \text{m}^{-3}$ a
$c = 334 \text{m} \cdot \text{s}^{-1}$. Zvuk reproduktoru působí na plochu
polokoule, proto $S = 2 \pi r^2$.

Pro zjistění akustického tlaku musíme převést hladinu akustického
tlaku na akustický tlak. To uděláme takto:

\[
  L_p = 20 \log \left( \frac{p}{p_0} \right) \ztoho
    p = 10^{\frac{L_p}{20}} \cdot p_0
\]

Protože hodnotu $p_0 = 20 \text{\textmu Pa}$ známe, stačí nám to všechno jen
dosadit do vzorce:

\[
  \eta = \frac{\frac{2 \pi r^2 \cdot 10^{\frac{L_p}{10}} \cdot p_0^2}{\rho c}}{P_e} =
    \frac{2 \pi r^2 \cdot 10^{\frac{L_p}{10}} \cdot p_0^2}{P_e \rho c}
\]

Ještě ale musíme zjistit vzdálenost $r$. Protože se budou sluchátka testovat
na umělých hlavách, budeme předpokládat, že vzdálenost bude rovna délce
ušního kanálku, proto $r = 2,5 \text{cm}$. Teď už známe všechny údaje
pro výpočet:

\[
  \eta = \frac{2 \pi \cdot (2,5 \cdot 10^{-2})^2 \cdot 10^{\frac{105}{10}} \cdot
    (20 \cdot 10^{-6})^2}{1,2 \cdot 334 \cdot 10^{-3}} \doteq 0,12 = 12 \%
\]

Proto účinnost těchto sluchátek je $12\%$.

\clearpage
\section{Problém 4}

Celkem jsem zanalyzoval účinnost 7 reproduktorů. Mezi nimi jsou sluchátka
a špunty, bezdrátový i drátový, bezdrátové reproduktory a normální bedny.
U každého reproduktoru jsem vzal její citlivost a z ní jsem za pomoci
vzorečku z předchozí úlohy vypočítal účinnost. U sluchátek jsem bohužel
musel jako v minulém cvičení odhadovat vzdálenost $r$, proto výsledná účinnost
pravděpodobně nebude přesně odpovídat skutečnosti. U špuntů jsem uvažoval
jen délku ušního kanálku a u sluchátek přes hlavu jsem ještě přidal centimetr
navíc. Níže už jsou výsledky:

\begin{table}[h!]
  \caption{Účinnost různých druhů reproduktorů}
  \label{tab:eff}
  \begin{center}
    \begin{tabular}{|c|c|c|c|c|c|}
      \hline
        Model & Typ & $L_p$ [dB] & $r$ [m] & $P$ [W] & $\eta$ [\%] \\
      \hline
        \href{https://www.harmanaudio.in/on/demandware.static/-/Sites-masterCatalog_Harman/default/dwd36fda61/pdfs/JBL_Tune_510BT_SpecSheet_English.pdf}{JBL TUNE 510BT} & Bluetooth přes hlavu & 103,5 & $3,5 \cdot 10^{-2}$ & $1,0 \cdot 10^{-3}$ & 17,19 \\
        \href{https://www.jbl.com/on/demandware.static/-/Sites-masterCatalog_Harman/default/dw2507ba12/pdfs/JBL_Tune_230NCTWS_SpecSheet_USA_English.pdf}{JBL Tune 230NC TWS} & Bluetooth špunty & 105 & $2,5 \cdot 10^{-2}$ & $1,0 \cdot 10^{-3}$ & 12,39 \\
        \href{https://www.marshallheadphones.com/cz/en/stanmore-ii-bluetooth.html}{Marshall Stanmore II} & Bluetooth reproduktor & 101 & 1 & 1 & 7,89 \\
        \href{https://www.sony.com/electronics/support/wired-headphones-headband/mdr-zx110/specifications}{Sony MDR-ZX110} & Drátové přes hlavu & 98 & $3,5 \cdot 10^{-2}$ & $1,0 \cdot 10^{-3}$ & 4,84 \\
        \href{https://mm.jbl.com/in-ear-headphones/JBL+T110.html}{JBL T110}  & Drátové špunty & 96 & $2,5 \cdot 10^{-2}$ & $1,0 \cdot 10^{-3}$ & 1,56 \\
        \href{https://reprosoustavy-reproduktory.heureka.cz/magnat-monitor-supreme-102/#specifikace/}{Magnat Monitor Supreme 102} & Bedna & 89 & 1 & 1 & 0,5 \\
        \href{https://reprosoustavy-reproduktory.heureka.cz/logitech-z200/#specifikace/}{Logitech Z200} & Bedna & 88 & 1 & 1 & 0,4  \\
      \hline
    \end{tabular}
  \end{center}
\end{table}

Jak šlo očekávat, bezdrátové reproduktory mají nejvyšší účinnost. Nejvyšší účinnosti
dosáhli bezdrátová sluchátka přes hlavu JBL TUNE 510BT, která dosáhla na
reproduktory až podezřele vysokých 17,2 \%. Tohle možná bude způsobeno tím,
že náš předpoklad, že $r = 3,5$ cm, je chybný, ale i přesto je jasné,
tyto sluchátka jsou velmi účinná.

Zároveň si můžeme všimnout toho jevu, že menší reproduktory jsou většinou 
účinnější než třeba bedny. To lze jednoduše vysvětlit -- pro většinu lidí,
kteří si kupují bedny, není účinnost priorita a narozdíl od beden je sluchátkám 
většinou dodávána energie přes 3,5 mm jack z počítače či telefonu, což 
omezuje velikost příkonu. Díky tomu se místo toho soustředí na kvalitu zvuku 
a šířce frekvenčního rozmezí na úkor účinnosti. 

Pokud se nezobrazují v tabulce linky na zdroje, jsou níže:

\begin{enumerate}
 \item JBL TUNE 510BT -- \url{https://www.harmanaudio.in/on/demandware.static/-/Sites-masterCatalog_Harman/default/dwd36fda61/pdfs/JBL_Tune_510BT_SpecSheet_English.pdf} 
 \item JBL Tune 230NC TWS -- \url{https://www.jbl.com/on/demandware.static/-/Sites-masterCatalog_Harman/default/dw2507ba12/pdfs/JBL_Tune_230NCTWS_SpecSheet_USA_English.pdf} 
 \item Marshall Stanmore II -- \url{https://www.marshallheadphones.com/cz/en/stanmore-ii-bluetooth.html} 
 \item Sony MDR-ZX110 -- \url{https://www.sony.com/electronics/support/wired-headphones-headband/mdr-zx110/specifications} 
 \item JBL T110 -- \url{https://mm.jbl.com/in-ear-headphones/JBL+T110.html}  
 \item Magnat Monitor Supreme 102 -- \url{https://reprosoustavy-reproduktory.heureka.cz/magnat-monitor-supreme-102/#specifikace/} 
 \item Logitech Z200 -- \url{https://reprosoustavy-reproduktory.heureka.cz/logitech-z200/#specifikace/} 
\end{enumerate}

\end{document}
