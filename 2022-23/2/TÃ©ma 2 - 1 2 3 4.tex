\documentclass{fkssolpub}

\usepackage[czech]{babel}
\usepackage{fontspec}

\usepackage{fkssugar}
 
\author{Ondřej Sedláček}
\school{Gymnázium Oty Pavla} 
\series{2}
\problem{2} 

\begin{document} 

\section{Úloha 1}

Korekcí hygienických limitů je mnoho -- některé platí pro 2 m okolo
obytných, nemocničních, vzdělávacích a rekreačních staveb, některé pro 
jejich interiér. Například pro interiér vzdělávacích staveb je korekce pro
hluk, s výjimkou hudby a řeči, oproti základnímu limitu pro interiér,
který je $L_{Aeq} = 40 \text{dB}$, +5 dB.

Avšak nejvýraznější kladná korekce, kterou jsem našel, je pro "starou hlukovou
zátěž", neboli pro silnice a železnice, které byli postaveny před rokem 2000. 
Pro ně je korekce v hluku 2 m okolo staveb kromě nemocnic, lázní a hospodářských 
staveb oproti základnímu limitu $L_{Aeq} = 50 \text{dB}$ až +20 dB v denní době.

\section{Úloha 2}

Myslím si, že by se stát měl pokoušet o to, aby i staré dopravní cesty byly
co nejméně hlučné. Proto bych limit pro "starou hlukovou zátěž" dával jedině
na žádost o výjimku a jen na omezenou dobu, aby byl nějaký tlak k rekonstrukci.
Přeci jenom je 70 dB hodně.

\section{Problém 3}

Pokud jsme si jistý, že hluk ze zdroje převyšuje hygienické limity, lze
jít právnickou cestou. To však není optimální, protože asi nebudeme schopni
určit, jestli to porušuje zákon, a daný zdroj hluku může mít z limitů výjimku
(prostě byrokracie).

Projektanti můžou používat takové materiály, které dobře izolují zvuk. Okna
by měla mít různou šířku skel a měla by mezi nimi být široká mezera.
Co se týče střech, nejlepší jsou pálené tašky. Podlaha také hraje roli, 
některé materiály (třeba vinyl) jsou při chůzi hlučnější než
ostatní. U konstrukce zdí se můžou rovnou použít dobře izolující materiály
(třeba akustické cihly) nebo se přidá izolace (třeba minerální či polystyren). 
Když ale nemáme to štěstí, lze izolace přidat do interiéru i dodatečně.

\section{Problém 4}

Abychom mohli vytvořit co největší hluk, musíme nějakým způsobem vytvořit
velkou práci, která se přemění na akustickou energii. Na to pravděpodobně
lidská síla stačit nebude. Ani gravitací si nepomůžeme (jedině, že bychom byli
schopni odvrátit nějaký velký meteorit). Ale přece můžeme naráz vypustit
velké množství energie za pomoci atomové bomby, např. car-bomby. 

Avšak ve vzduchu existuje limit, kdy se bere zvuková vlna jako zvuková,
a to v nulové nadmořské výšce 194 dB. Tehdy je akustický tlak tak vysoký,
že mezi vlnami začne vznikat vakuum, proto u vyšších hodnot už je vlna spíš
rázová. Proto ve vzduchu lze dosáhnout jen hladiny intenzity 194 dB, čehož zrovna 
car-bomba dosáhla.

Ve vodě je však možné dosáhnout ještě větší hladiny intenzity zvuku, což je
hodně ovlivněno tím, že prahový akustický tlak v oceánu je nižší než na souši.
Například u vorvaně bylo zaznamenáno až 230 dB. Proto, když odpálíme bombu pod
vodou, mohli bychom dosáhnout větší hladiny intenzity.

\end{document}
