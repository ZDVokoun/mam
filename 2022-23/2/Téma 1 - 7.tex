\documentclass{fkssolpub}

\usepackage[czech]{babel}
\usepackage{fontspec}

\usepackage{fkssugar}
 
\author{Ondřej Sedláček}
\school{Gymnázium Oty Pavla} 
\series{1}
\problem{7} 

\begin{document} 

\section{Úloha 7}

$\exists$, $\not\exists$ král

Byl, nebyl jeden stát, jemuž vládl jeden vůdce. Ten byl velmi mocný,
avšak i zlý a velmi krutý. Jednou za ním přišel jeho rádce, řka:

"Nejvyšší fíra, už jsme porazili všechny až na jedinou zemi -- na
ty prcky víly. Ano, jen kvuli těm smradům. Chce to pořádný fajty,
ale před tim musime zjistit, jak moc jsou silni. Proto vyhlaš, ať každej, 
kdo má ňáké info, je pěkně nám vyklopí, nebo chcípne s celou svou
rodinou."

Vůdce se na okamžení zamyslel, pak povelil:

"OK, provedu. Odchod!"

Avšak velmi daleko na venkově o tom Vendelín neměl ani potuchy. Ten
měl úplně jiné starosti.

"Vendelínku, už moc stará a slabá jsem a ty's mladý a plný síly. Nejsem
schopna se o tebe víc postarat. Vezmi si, co potřebuješ, a vyraž do
světa."

Když už byl na odchodu, řekla mu ještě:

"Ještě tu máš buchty, abys neměl hlad."

"Děkuji, maminko." děl a kráčel přímo za nosem.

Několik dní a nocí procházel světem a hledal místo k žití. Jednou ale
dorazil k černému a hustě zarostlému lesu. Vendelína najednou chytla 
zvědavost a protože si neuvědomil nebezpečí jeho činu, pokusil se 
tím lesem projít. Protože neměl sebou nic ostrého kromě rybičky, šlo
mu projít jen stěží. Po noci stráveném prolízáním se však celý 
poškrábaný a od krve objevil na druhé straně.

Tento typ krajiny ještě nikdy v životě neviděl. Bylo ještě brzo ráno,
tudíž mohl vidět v dáli mnoho bílých světýlek z domovů. Byli slyšet
zvuky tekoucí vody, cvrčků, malých ptáčků. A viděl skoro jako za bílé
noci -- slabě bíle svítila voda, světlušky, stromy a kořeny slabě 
zeleně.

Když najednou ucítil na spánku úder a spadl na zem. Chvíli nic neviděl,
jen nějaké bílé světlo. Pak se vzpamatoval a uviděl před sebou krásnou
malou dívku s malými křídly, vílu.

S překvapeným výrazem a rukou připravenou k útoku na něj hleděla.
Když si prohlídla jeho celé tělo, zjistila, že je bezbranný. V tom
se na jejím obličeji objevil starostlivý výraz.

"Odpusť mi. Jsem Sofie. Ztratil ses?"

On na ni zmateně hleděl. Je to sen? Vůbec nevěřil svým očím. A taky
nerozuměl ani slovu, což víle brzy došlo a naznačila mu, ať čeká.

Přilítla potom s jednou velkou knihou a podala mu ji. Byla to učebnice.
Přestože jediné, co četl, byly knihy pro děti, dokázal pak složit
větu.

"Jsem Vendelín."

To ji očividně pobavilo a on se díval na její krásný úsměv. Pak
si všimla svítajícího slunce a zděsila se.

"Musíš rychle jít, Vendelíne. Večer se zase vrať. Budu čekat," a 
odletěla.

Když se vrátil zpět do lidského světa, bylo mu jasné, že když
nemůže žít ve vílím světě, zůstane alespoň u hranic. Našel si práci u
sedláka a ve volném čase se učil vznešené řeči víl.

Jednou však musel jít do města, aby koupil nové náčiní. Když se vracel, 
všiml si jedné cedule červené, jež dila: 

\textit{"Dle nařízení nejvyššího vládce musí všichni veškere info vo
vílách vyklopit, nebo chcípne s celou svou rodinou!! Spolupracovníka
čeká štědrá odměna."}

Protože Vendelín věděl od maminky, že vládci nikdy nemají dobré úmysly,
běžel co nejrychleji do světa víl, aby ji upozornil:

"Sofie, musím ti něco říct," řekl a vzal ji za ruku. "Vládce lidského
světa chce po lidech, aby mu řekli vše o vílách, jinak budou zabiti."
Víla si uvědomila, oč tu běží. Hrozí válka.



\end{document}
