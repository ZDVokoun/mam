\documentclass{fkssolpub}

\usepackage[czech]{babel}
\usepackage{fontspec}

\usepackage{fkssugar}

% \usepackage{tikz}
\usepackage{pgfplots}
 
\author{Ondřej Sedláček}
\school{Gymnázium Oty Pavla} 
\series{2}
\problem{4} 

\begin{document} 

\section{Úloha 1}

\begin{equation}
  \label{eq:1}
  (x^2 - 4x + 4)' = (x^2)' - (4x)' + (4)' = 2x - 4 = 2 \cdot (x - 2)
\end{equation}

\begin{equation}
  \label{eq:2}
  (x^3 - 6x^2 + 12x - 8)' = (x^3)' - (6x^2)' + (12x)' - (8)'
   = 3x^2 - 6 \cdot 2x + 12 = 3 \cdot (x^2 - 4x + 4)
\end{equation}

\begin{equation}
  \label{eq:3}
  (2 \sin{x} \cos{x})' = 2 \cdot (\sin{x} \cos{x})' 
   = 2 \cdot ((\sin{x})' \cdot \cos{x} + \sin{x} \cdot (\cos{x})')
   = 2 \cdot (\cos^2 x - \sin^2 x) = 2 \cdot \cos{2x}
\end{equation}


\section{Úloha 2}

U funkce \ref{eq:1} je toto jednoduché, protože její derivace je 
prostou lineární rovnici, která protíná osu x jen v jediném bodě,
neboli hodnota $x$, která vynuluje $y$, což je $x = 2$. A protože 
tato funkce je rostoucí (koeficient $a$ je kladný),
funkce \ref{eq:1} roste v intervalu $\langle 2;+\infty)$ a klesá v 
intervalu $(-\infty;2 \rangle$.

Derivace funkce \ref{eq:2} je funkcí kvadratickou, tudíž budeme muset
zjistit její kořeny. V tomto případě nebude potřeba použít diskriminant,
použijeme místo něj vzorec pro umocněný rozdíl:

\[
  3 \cdot (x^2 - 4x + 4) = 3 (x - 2)^2
\]

Jak lze vidět, jedná se o kvadratickou funkci s dvojnásobným kořenem, tudíž
funkce \ref{eq:2} musí růst neustále.

Teď nám zbývá jen funkce \ref{eq:3}. Protože se jedná o periodickou
funkci, budu určovat interval klesání s ohledem na $k \in \mathbb{Z}$.

Abychom určili chování derivace, níže je tabulka:

\begin{table}[h]
  \caption{Tabulka hodnot derivace}
  \label{tab:1}
  \begin{center}
    \begin{tabular}{|c|c|c|c|c|}
      \hline
      $x$ & $0$ & $\frac{\pi}{4}$ & $\frac{\pi}{2}$ & $\frac{3\pi}{4}$ \\
      \hline
      $2 \cdot \cos{2x}$ & $2$ & $0$ & $-2$ & $0$ \\
      \hline
    \end{tabular}
  \end{center}
\end{table}

Z této tabulky lze vidět, že v intervalu $\langle 0; \pi \rangle$ klesá
funkce \ref{eq:3} v intervalu $\langle \frac{\pi}{4}; \frac{3\pi}{4} \rangle$ 
a roste v intervalu $\langle 0; \frac{\pi}{4} \rangle 
\cup \langle \frac{3\pi}{4}; \pi \rangle$. Tohle, když zobecníme, vyjde nám,
že funkce \ref{eq:3} roste v intervalu $\langle -\frac{\pi}{4} + k\pi; \frac{\pi}{4} + k\pi \rangle$
a klesá v intervalu $\langle \frac{\pi}{4} + k\pi; \frac{3\pi}{4} + k\pi \rangle$

\begin{tikzpicture}
  \begin{axis} [
    axis lines=center,
    xtick distance=2.0,
    ymax=5,
    ymin=-5,
    xmax=5,
    xmin=-5,
    domain=-5:5,
    legend cell align={left},
    legend pos=north west
  ]
    \addplot [blue,smooth,thick] { x^2 - 4*x + 4 };
    \addplot [red,smooth,thick] { x^3 - 6*x^2 + 12*x - 8 };
    \addplot [teal,smooth,thick,samples=200] { 
      2*sin(deg(x))*cos(deg(x))
    };
    \legend{
      První funkce,
      Druhá funkce,
      Třetí funkce
    }
  \end{axis}
\end{tikzpicture}

\section{Úloha 3}

Protože extrémy jsou krajními body intervalů růstu a klesání, stačí se nám
podívat na intervaly z druhé úlohy, abychom získali extrémy funkcí.

U funkce \ref{eq:1} nám stačí určit, kdy její derivace, lineární funkce, 
protíná osu $x$, a to v bodě $x = 2$. Neboť derivace je rostoucí, jedná
se globální minimum.

U funkce \ref{eq:2} můžeme chybně určit, že bod $x = 2$ bude extrémem,
ale protože se jedná o dvojnásobný kořen její derivace a samotná funkce
roste v $\mathbb{R}$, nemůže to být extrém. Tudíž funkce \ref{eq:2}
nemá žádný extrém.

A poněvadž je funkce \ref{eq:3} periodická, pokud bude mít globální
maxima a minima, bude jich mít nekonečně mnoho, což musí mít, anžto
existují intervaly, kdy roste a kdy klesá. Když se podíváme na 
intervaly růstu a klesání dané funkce, zjistíme, že pro 
$k \in \mathbb{Z}$ budou globální maxima $\frac{\pi}{4} + k\pi$ 
a globální minima $\frac{3\pi}{4} + k\pi$.

\end{document}
